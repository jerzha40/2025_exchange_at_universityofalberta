\documentclass{article}
% \documentclass[UTF8]{ctexart}
\usepackage{amsmath,amssymb,amsfonts}  % For math symbols and fonts
\usepackage{graphicx}                   % For including images
\usepackage{hyperref}                   % For hyperlinks
\usepackage{cite}                       % For citations
\usepackage[UTF8]{ctex}  % 注意,不是 \documentclass[UTF8]{ctexart}
\renewcommand{\contentsname}{Contents}
\renewcommand{\refname}{References}
\renewcommand{\abstractname}{Abstract}

\usepackage{algorithm}
\usepackage{algorithmic}

\usepackage{amsthm}
% Define new theorem-like environments
% Define environments
\theoremstyle{definition} % Non-italicized style
\newtheorem{definition}{Definition}[section]
\newtheorem{exercise}{Exercise}[section]

% Title and author info
\title{Discretized ODE PINN Solver}
\author{
Zhang Jinrui\thanks{alternative email:zhangjr1022@mails.jlu.edu.cn} \\ \texttt{jerryzhang40@gmail.com}
}

\date{20250720}  % Empty date; optional, you can also specify a date here

\begin{document}

\maketitle

\begin{abstract}
    In this article, I tried to make
    a PINN\cite{raissi2017physics} solver for discretized
    ode problem.
\end{abstract}

\section{state the problem}
The general autonomous system can written
in the following form.
\[
    \dot{x}=f(x)
\]
where generally \(x\in\mathbb{R}^n\)

This is a dynamical system where the
underlying Space is \(\mathbb{R}^n\).
If denote \(\Phi(x,t)=\Phi_t(x)\),
then the group is \(t\in\mathbb{R}\),
equiped with the group operation
\(\Phi_{t_2}(\Phi_{t_1}(x))=\Phi_{t_1+t_2}(x)\).

The discretized by a constant time
version of this problem is by setting a
constant time interval and turn this
continuous problem on \(\mathbb{R}\)
to \(\mathbb{Z}\).
If the discretized time is \(\Delta\)
then the group of this system is
\(\mathbb{Z}\) and the group operation
is
\(\Phi_{m\Delta}(\Phi_{n\Delta}(x))
=
\Phi_{(n+m)\Delta}(x)\).

So the only thing we need to find
for this discretized problem is
\(\Phi(x,\Delta)\) which is the
unit of the group \(\mathbb{Z}\)

\section{derive the pde relations}
For the time interval \([0,\Delta]\),
The continuous problem will have
partial derivatives for the solution
\(\Phi(x,t)\).

Which is as follow.
\[
    \frac{\partial\Phi}{\partial t}=f(\Phi)
\]
and
\[
    \frac{\partial^2\Phi}{\partial x\partial t}
    =
    f^\prime(\Phi)\frac{\partial\Phi}{\partial x}
\]

\section{road map}
The basic idea is to solve the pde in the
previous section on \((0,\Delta)\),
to get the \(\Phi(x,\Delta)\).

For more continuous solution
\(\Phi((x,t))\), we just need to find the
largest \(n\) such that \(n\Delta\leq t\),
then use \(\Phi^n_\Delta(x)=\Phi(x,n\Delta)\)
and then for the system is a autonomous system
we can just apply one more transform,
\(\Phi(x,t)=\Phi_{t-n\Delta}(\Phi_{n\Delta}(x))\).
The function \(\Phi(x,t-n\Delta),t-n\Delta\in[0,\Delta)\)
is already solved at the first time.

We can then obtain \(\Phi_{2^k\Delta}\)
by repeatly take the function composition,
and for each \(k\) we can
continue to train \(\Phi_{2^k\Delta}\)
base on \(\Phi_\Delta\) to get more accuracy.

\section{residual model}
To learn the \(\Phi_{\Delta}(x)\), we basically
need to learn a \(\mathbb{R}\to\mathbb{R}\)
function which is very hard to maintain for computer to
achieve.

And for all function the initial value \(\Phi_{0}(x)=x\),
we may only learn the difference function
\(\delta(x,t)=\Phi(x,t)-x\) or
\(\delta_t(x)=\Phi_t(x)-x\).

Then the differential equation
is like the following.
\[
    \frac{\partial\delta}{\partial t}=f(\delta+x)-x
\]
and
\[
    \frac{\partial^2\delta}{\partial x\partial t}
    =
    f^\prime(\delta+x)\frac{\partial\delta}{\partial x}
\]

\section{residual euler model}
To lean some effort from the conventional numerical scheme
I want to use some euler method.

Separation-----------------

There are 10 drones and fly on the sky
obeys Newton's second law of motion.
which is
\[
    \vec{F} = m \vec{a}
\]
\[
    \vec{a} = \frac{d\vec{v}}{dt} = \frac{d^2 \vec{x}}{dt^2}
\]
And I mean the policy by, we need a function of force
depending on some communication between drones
to decide the \(\vec{F}\)
\[
    \vec{F}=f(the current information)
\]

And then we want the following dynamic system
\[\begin{bmatrix}
        \frac{d\vec{d}}{dt} \\
        \frac{d\vec{v}}{dt}
    \end{bmatrix}=
    \begin{bmatrix}
        \vec{v} \\
        \vec{a}=f/m
    \end{bmatrix}
\]
has some Self-organized emergent phenomena,
to automatically emergent a circle rounding pattern.

\section{Jinrui Zhang's prompt}
\subsection{a simple prompt}
To be more clear of the notations we use,
we have \(i\in\{1,2,...,10\}=N\)
And the drones are ignored of its flying
height, which the position vector can be a
2d vector note it as \(\vec{d_i}\)
And so the velocity and acceleration we denote as
\(\vec{v_i}=\frac{d\vec{d_i}}{dt}\)
and
\(\vec{a_i}=\frac{d\vec{v_i}}{dt}\)
I want to prompt a \(f\) so that it can form
a cirle.
\[
    \vec{f_i}=m_i(\sum_{\forall k\neq i,\|\vec{d_i}-\vec{d_k}\|\leq R}(\frac{\vec{d_i}-\vec{d_k}}{\|\vec{d_i}-\vec{d_k}\|^3})+(\frac{\vec{d_{t(i)}}-\vec{d_i}}{\|\vec{d_{t(i)}}-\vec{d_i}\|}-v_i))
\]
This model is easy to explain, the first term is
just a inverse square propell force, the second
term is make the velocity quickly approch a set direction
the \(t(i)\) is just a randomly choosed target drone
other than \(i\) that is \(t(i)\in N, t(i)\neq i\)

This formula can be rewrite without physical term as
follow.
\[
    \vec{a_i}=\sum_{\forall k\neq i,\|\vec{d_i}-\vec{d_k}\|\leq R}(\frac{\vec{d_i}-\vec{d_k}}{\|\vec{d_i}-\vec{d_k}\|^3})+(\frac{\vec{d_{t(i)}}-\vec{d_i}}{\|\vec{d_{t(i)}}-\vec{d_i}\|}-v_i)
\]
separately view this is combined by two independent force
\[
    (\vec{a_i})_{target}=(\frac{\vec{d_{t(i)}}-\vec{d_i}}{\|\vec{d_{t(i)}}-\vec{d_i}\|}-v_i)
\]
\[
    (\vec{a_i})_{propell}=\sum_{\forall k\neq i,\|\vec{d_i}-\vec{d_k}\|\leq R}(\frac{\vec{d_i}-\vec{d_k}}{\|\vec{d_i}-\vec{d_k}\|^3})
\]

\subsection{a simple prompt:simulation}
\subsubsection{Four drone case:derivation}
This case just choose \(N=\{1,2,3,4\}\)
and don't allow \(t(t(i))=i\)
which definitely form a three element loop and
a dangling drone.

We have a (4,2)-tensor \(\vec{d_i}\)
and two other (4,2)-tensor \(\vec{v_i}\)
and \(\vec{a_i}\)
The initial points are randomly choosed
in Uniformly \([0,1]\times[0,1]\)

choose a time increment \(dt\)
and the simulation update formula is simple to write

just as follow
\[\begin{bmatrix}
        \vec{{d_{n+1}}_i} \\
        \vec{{v_{n+1}}_i}
    \end{bmatrix}=
    \begin{bmatrix}
        \vec{{d_{n}}_i}+\vec{{v_n}_i}dt \\
        \vec{{v_{n}}_i}+(\sum_{\forall k\neq i,\|\vec{{d_n}_i}-\vec{{d_n}_k}\|\leq R}(\frac{\vec{{d_n}_i}-\vec{{d_n}_k}}{\|\vec{{d_n}_i}-\vec{{d_n}_k}\|^3})+(\frac{\vec{{d_n}_{t(i)}}-\vec{{d_n}_i}}{\|\vec{{d_n}_{t(i)}}-\vec{{d_n}_i}\|}-{v_n}_i))dt
    \end{bmatrix}
\]
simple Euler method.

\subsubsection{Four drone case:code \& result}
% the computational code are \cite[FourDroneCase]{FourDroneCase} .
% the results are shown by the following pictrues which
% generated by the code.
% The video are \cite[sample1-video]{sample1-video}
% and \cite[sample2-video]{sample2-video}
% and more other in the same folder on github.
% \begin{figure}[ht!]
%     \centering
%     \begin{minipage}{0.45\textwidth}
%         \centering
%         \includegraphics[width=0.9\textwidth]{fig/sample1/dd.png} % first figure itself
%         \caption{sample1 randomly initial position}
%         \label{fig:fig1}
%     \end{minipage}\hfill
%     \begin{minipage}{0.45\textwidth}
%         \centering
%         \includegraphics[width=0.9\textwidth]{fig/sample1/dd_18254.png} % second figure itself
%         \caption{sample1 after a period of time}
%     \end{minipage}
% \end{figure}
% \begin{figure}[ht!]
%     \centering
%     \begin{minipage}{0.45\textwidth}
%         \centering
%         \includegraphics[width=0.9\textwidth]{fig/sample2/dd.png} % first figure itself
%         \caption{sample1 randomly initial position}
%         \label{fig:fig2}
%     \end{minipage}\hfill
%     \begin{minipage}{0.45\textwidth}
%         \centering
%         \includegraphics[width=0.9\textwidth]{fig/sample2/dd_18993.png} % second figure itself
%         \caption{sample2 after a period of time}
%     \end{minipage}
% \end{figure}

\subsubsection{Ten drone case:derivation}
undergoing

\subsubsection{Ten drone case:result}
undergoing

\subsection{some analysis why it will have a stability property}
\subsubsection{the terminate radius R}
undergoing

\subsubsection{the terminate center O}
undergoing

\subsubsection{graph theory part}
The \(t(i)\) forms a graph which have \(n\) points
and \(n\) oriented edges, this forms a tree with a extra
edges, and this case It will obviously form a Unicyclic Graph.

Which is a tree if we treat all the point on the
loop as the same point.

\subsection{target distance method}
undergoing

\subsection{target distance method:simulation}
undergoing

\section{Zinan Su's approach}
\subsection{notations \& equations}
safe collide radius is \(d_s\)
\[
    \sigma=2d_s
\]
\(NUM\) is the total number of the drones.
And then we want the following dynamic system
\[\begin{bmatrix}
        \frac{d\vec{p_i}}{dt} \\
        \frac{d\vec{v_i}}{dt}
    \end{bmatrix}=
    \begin{bmatrix}
        \vec{v_i} \\
        \vec{a_i}
    \end{bmatrix}
\]
circle origin is a function
\[
    c=\frac{1}{NUM}\sum_{k=1}^{NUM}p_k
\]
Four constants.
\[
    k_p=
\]
\[
    k_d=
\]
\[
    k_v=
\]
\[
    k_r=
\]
\[
    R^*=\frac{1}{NUM}\sum_{k=1}^{NUM}p_k(0)-c(0)
\]
\[
    v_d=\frac{1}{NUM}\sum_{k=1}^{NUM}\|v_k(0)\|
\]
and
\[
    r_i=p_i-c
\]
\[
    d_i=\|r_i\|
\]
\[
    \hat{r_i}=\frac{r_i}{d_i}
\]
\[
    \hat{\theta_i}=\mathbb{M}_\theta r_i
\]
\[
    \mathbb{M}_\theta=
    \begin{bmatrix}
        0  & 1 \\
        -1 & 0
    \end{bmatrix}
\]
\[
    {v_{i}}_{\parallel}=\hat{r_i}\cdot v_i
\]
\[
    {v_{i}}_{\perp}=\hat{\theta_i}\cdot v_i
\]
\[
    U(r)=k_re^{-\frac{r}{2\sigma^2}}
\]
\[
    U_{ij}=U(\|p_i-p_j\|)
\]
\[
    \vec{u_i}_1=[-k_p(d_i-R^*)-k_d{v_{i}}_{\parallel}]\hat{r_i}
\]
\[
    \vec{u_i}_2=[-k_v({v_{i}}_{\perp}-v_d)]\hat{\theta_i}
\]
\[
    \vec{u_i}_3=\sum_{\forall k\neq i}(-\nabla_{p_i}U_{ij})
\]
\[
    \vec{u_i}=\vec{u_i}_1+\vec{u_i}_2+\vec{u_i}_3
\]

\subsection{analysis}
考虑带阻力的动力方程
\begin{equation*}
    m\frac{\mathrm{d}^2\boldsymbol{p}_i}{\mathrm{d}t^2}+c\left\|\frac{\mathrm{d}\boldsymbol{p}_i}{\mathrm{d}t}\right\|\frac{\mathrm{d}\boldsymbol{p}_i}{\mathrm{d}t}=\boldsymbol{u}_i,
\end{equation*}
其中 $m$ 为质量, $c$ 为阻力系数, 即
\begin{align}
    \frac{\mathrm{d}\boldsymbol{X}}{\mathrm{d}t}=\boldsymbol{F}(\boldsymbol{X}),
\end{align}
其中
\begin{equation*}
    \boldsymbol{X}=\binom{\boldsymbol{p}}{\boldsymbol{v}}\in\mathbb{R}^{40},\ \boldsymbol{p}=\begin{pmatrix}
        \boldsymbol{p}_1 \\\vdots\\\boldsymbol{p}_{10}
    \end{pmatrix},\ \boldsymbol{v}=\begin{pmatrix}
        \boldsymbol{v}_1 \\\vdots\\\boldsymbol{v}_{10}
    \end{pmatrix},\ \boldsymbol{F}(\boldsymbol{X})=\binom{\boldsymbol{v}}{m^{-1}(\boldsymbol{u}-c\|\boldsymbol{v}\|\boldsymbol{v})}.
\end{equation*}
易知质心 $\boldsymbol{c}(\boldsymbol{X}),\|\boldsymbol{r}_i\|,\widehat{\boldsymbol{r}_i},\mathit{\Phi}$ 均为 Lipschitz 连续函数, 且初值条件满足
\begin{equation*}
    \min_{i\ne j}\|\boldsymbol{p}_i(0)-\boldsymbol{p}_j(0)\|>d_s>0.
\end{equation*}
由 Picard 定理可知方程 (1) 的解存在且唯一, 解为
\begin{equation*}
    \boldsymbol{X}(t)=\boldsymbol{X}(0)+\int_0^t\boldsymbol{F}(\boldsymbol{X}(s))\mathrm{d}s,
\end{equation*}
即
\begin{align*}
     & \boldsymbol{v}_i(t)=\boldsymbol{v}_i(0)+\int_0^t\left[m^{-1}\boldsymbol{u}_i(\boldsymbol{X}(s))-\frac{c}{m}\|\boldsymbol{v}_i(s)\|\boldsymbol{v}_i(s)\right]\mathrm{d}s, \\
     & \boldsymbol{p}_i(t)=\boldsymbol{p}_i(0)+\int_0^t\boldsymbol{v}_i(s)\mathrm{d}s.
\end{align*}
下面证明防撞性. 定义
\begin{align*}
     & \mathit{\Psi}(d)=k_r\exp\left\{-\frac{(d-d_s)^2}{2\sigma^2}\right\},                                                                      \\
     & \mathit{\Phi}(d)=-\frac{\mathrm{d}\mathit{\Psi}}{\mathrm{d}d}=\frac{k_r}{\sigma^2}\exp\left\{-\frac{(d-d_s)^2}{2\sigma^2}\right\}(d-d_s), \\
     & E_{ij}(t)=\frac{1}{2}\left(\frac{\mathrm{d}d_{ij}}{\mathrm{d}t}\right)^2+\psi(d_{ij}(t)),
\end{align*}
其中 $d_{ij}(t)=\|\boldsymbol{p}_i(t)-\boldsymbol{p}_j(t)\|.$ 则系统能量为
\begin{equation*}
    \mathcal{E}(t)=\sum_{1\leqslant i<j\leqslant N}E_{ij}(t).
\end{equation*}
而
\begin{equation*}
    \frac{\mathrm{d}E_{ij}}{\mathrm{d}t}=\frac{\mathrm{d}d_{ij}}{\mathrm{d}t}\cdot\frac{\mathrm{d}^2d_{ij}}{\mathrm{d}t^2}+\mathit{\Phi}(d_{ij})\frac{\mathrm{d}d_{ij}}{\mathrm{d}t},
\end{equation*}
\begin{equation}
    \frac{\mathrm{d}^2d_{ij}}{\mathrm{d}t^2}=\frac{1}{d_{ij}}\left[\|\boldsymbol{v}_i-\boldsymbol{v}_j\|^2+(\boldsymbol{p}_i-\boldsymbol{p}_j)\cdot(\boldsymbol{u}_i-\boldsymbol{u}_j)-\left(\frac{\mathrm{d}d_{ij}}{\mathrm{d}t}\right)^2\right],
\end{equation}
其中 $\boldsymbol{u}_i=\dot{\boldsymbol{v}_i}.$ 对于 $\boldsymbol{u}_i$, 有
\begin{align*}
    \boldsymbol{u}_i=\frac{1}{m}[-k_p(d_i-R^*)\widehat{\boldsymbol{r}_i}-k_dv_{r,i}\widehat{\boldsymbol{r}_i}-k_v(v_{\theta,i}-v_d)\hat{\boldsymbol{\theta_i}}]+\frac{1}{m}\sum_{k\ne i}\mathit{\Phi}(d_{ik})(\boldsymbol{p}_i-\boldsymbol{p}_k)=:\boldsymbol{u}_{i_1}+\boldsymbol{u}_{i_2}.
\end{align*}
设 $\|\boldsymbol{u}_{i_1}-\boldsymbol{u}_{j_1}\|\leqslant L.$ 取
\begin{equation*}
    k_r>L\sigma^2\mathrm{e}^{\frac{1}{2}}\max\left\{\frac{1}{d_s},\frac{1}{\displaystyle\min_{k\ne l}d_{kl}(0)}\right\},
\end{equation*}
则当 $d_{ij}\leqslant d_s+\sigma$ 时, 有
\begin{equation*}
    \|\boldsymbol{u}_{i_2}-\boldsymbol{u}_{j_2}\|>2L.
\end{equation*}
于是
\begin{equation*}
    (\boldsymbol{p}_i-\boldsymbol{p}_j)\cdot(\boldsymbol{u}_i-\boldsymbol{u}_j)\geqslant\|\boldsymbol{u}_{i_2}-\boldsymbol{u}_{j_2}\|d_{ij}-Ld_{ij}>Ld_{ij}.
\end{equation*}
代入 (2) 式有
\begin{equation*}
    \frac{\mathrm{d}E_{ij}}{\mathrm{d}t}\geqslant\frac{\mathrm{d}d_{ij}}{\mathrm{d}t}(L+\mathit{\Phi}(d_{ij}))>0.
\end{equation*}
由此即知
\begin{equation*}
    \frac{\mathrm{d}\mathcal{E}}{\mathrm{d}t}\geqslant-\kappa\mathcal{E}(t),
\end{equation*}
其中 $\kappa>0$ 为常数. 而
\begin{align*}
     & \mathcal{E}(0)\geqslant\sum_{i<j}\mathit{\Psi}(d_{ij}(0))>\psi(d_s+\sigma)\cdot\binom{N}{2}, \\
     & \mathcal{E}(t)\geqslant\mathcal{E}(0)\mathrm{e}^{-\kappa t}>0,                               \\
     & \mathit{\Psi}(d_{ij}(t))\leqslant E_{ij}(t)\leqslant\mathcal{E}(t).
\end{align*}
由 $\mathit{\Psi}$ 严格单调递减可知
\begin{equation*}
    d_{ij}(t)\geqslant\mathit{\Psi}^{-1}(\mathcal{E}(t))>\mathit{\Psi}^{-1}(\mathcal{E}(0)\mathrm{e}^{-\kappa t}).
\end{equation*}
令 $t\to\infty,$ 有
\begin{equation*}
    \varliminf_{t\to\infty}d_{ij}(t)\geqslant\mathit{\Psi}^{-1}(0)=d_s.
\end{equation*}
故总是不会相撞.\par
下面讨论收敛性, 即讨论系统收敛至
\begin{equation*}
    \mathcal{S}=\{\|\boldsymbol{r}_i\|=R,\boldsymbol{v}_i\cdot\widehat{\boldsymbol{r}_i}=0,\|\boldsymbol{v}_i\|=v_d\}.
\end{equation*}
构造 Lyapunov 函数
\begin{equation*}
    V=\frac{1}{2}\sum_{i=1}^N[k_p(d_i-R)^2+\|\boldsymbol{v}_i-v_d\widehat{\boldsymbol{\theta}_i}\|^2]+\sum_{i<j}\mathit{\Psi}(d_{ij}),
\end{equation*}
则
\begin{equation*}
    \dot{V}=-\sum_{i}k_dv_{r,i}^2-\sum_ik_v(v_{\theta,i}-v_d)^2-\sum_ic\|\boldsymbol{v}_i\|^3\leqslant0,
\end{equation*}
故方程渐进收敛至 $\mathcal{S}$.

\subsection{some constants calculation}
undergoing
\(c\) is air resistance constant.
\[
    m\frac{d^2 \vec{p_i}}{dt^2}+c\|\frac{d\vec{p_i}}{dt}\|\frac{d\vec{p_i}}{dt}=u_i
\]



\bibliographystyle{plain}  % or another style like unsrt, IEEEtran, etc.
\bibliography{references}  % references.bib is the file name

\end{document}
