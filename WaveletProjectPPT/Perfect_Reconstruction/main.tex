\documentclass[11pt]{beamer}
\usetheme{Madrid}
\usecolortheme{default}
\usepackage{amsmath,amssymb,amsthm}
\usepackage{geometry}
\usepackage{graphicx}
\setbeamertemplate{navigation symbols}{}

\title[Framelet Transforms]{Perfect Reconstruction of\\ Discrete Framelet Transforms}
\subtitle{Chapter 1.1.2 Summary}
\author{Based on \emph{Framelets and Wavelets}\\ Bin Han}
\date{\today}

\begin{document}

\begin{frame}
  \maketitle
\end{frame}

\section{Basic Definitions}

\begin{frame}{Sequence Spaces}
  \begin{itemize}
    \item $\ell(\mathbb{Z})=\bigl\{v=\{v(k)\}_{k\in\mathbb{Z}}\mid v(k)\in\mathbb{C}\bigr\}$
    \item $\ell_0(\mathbb{Z})=\bigl\{u\in\ell(\mathbb{Z})\mid \text{only finitely many } u(k)\neq 0\bigr\}$
    \item filter bank: $\{u_0,\dots,u_s\}\subset \ell_0(\mathbb{Z})$
  \end{itemize}
\end{frame}

\section{Key Operators}

\begin{frame}{Operators}
  \begin{itemize}
    \item \textbf{Subdivision:}
          \[
            [S_u v](n)=2\sum_{k\in\mathbb{Z}} v(k)\,u(n-2k)
          \]
    \item \textbf{Transition:}
          \[
            [T_u v](n)=2\sum_{k\in\mathbb{Z}} v(k)\,\overline{u(k-2n)}
          \]
    \item \textbf{Discrete Framelet Transform (DFrT):}
          \[
            \frac12\sum_{\ell=0}^{s} S_{u_\ell}T_{\tilde u_\ell}v
          \]
  \end{itemize}
\end{frame}

\section{Perfect Reconstruction (PR)}

\begin{frame}{Perfect Reconstruction Condition}
  \begin{itemize}
    \item the transform satisfies perfect reconstruction if
          \[
            \frac{1}{2}\sum_{\ell=0}^s S_{u_\ell} T_{\tilde{u}_\ell} v = v \quad \forall v\in\ell(\mathbb{Z}).
          \]
  \end{itemize}
\end{frame}

\section{Equivalent Characterizations}

\begin{frame}{Main Theorem -- Four Equivalent Statements}
  The following are equivalent:

  \begin{enumerate}
    \item The filter bank $(\{\tilde u_\ell\},\{u_\ell\})$ has PR.
    \item The identity $\displaystyle\frac12\sum_{\ell=0}^{s} S_{u_\ell}T_{\tilde u_\ell}v = v$ holds for all $v\in\ell_0(\mathbb{Z})$.
    \item The identity $\displaystyle\frac12\sum_{\ell=0}^{s} S_{u_\ell}T_{\tilde u_\ell}v = v$ holds for $v=\delta$ and $v=\delta(\cdot-1)$.
    \item Frequency-domain identities for all $\omega\in\mathbb{R}$:
          \[
            \sum_{\ell=0}^{s}\overline{\hat{\tilde u}_\ell(\omega)}\hat u_\ell(\omega)=1,
          \]
          \[
            \sum_{\ell=0}^{s}\overline{\hat{\tilde u}_\ell(\omega)}\hat u_\ell(\omega+\pi)=0.
          \]
  \end{enumerate}
\end{frame}

\section{Proof Sketch}

\subsection*{Implications}

\begin{frame}{(i) $\Rightarrow$ (ii) $\Rightarrow$ (iii)}
  Trivial inclusions:
  \[
    \ell_0(\mathbb{Z})\subseteq \ell(\mathbb{Z})\quad\text{and}\quad \delta,\delta(\cdot-1)\in\ell_0(\mathbb{Z}).
  \]
\end{frame}

\begin{frame}{(iii) $\Rightarrow$ (iv)}
  \begin{enumerate}
    \item \textbf{Frequency Domain Conversion:} Using Fourier transforms of $T_{\tilde{u}_\ell}$ and $S_{u_\ell}$:
          \[
            \widehat{T_{\tilde{u}_\ell} v}(\omega) = \hat{v}(\omega/2)\overline{\hat{\tilde{u}}_\ell(\omega/2)} + \hat{v}(\omega/2+\pi)\overline{\hat{\tilde{u}}_\ell(\omega/2+\pi)},
          \]
          \[
            \widehat{S_{u_\ell} w}(\omega) = \hat{w}(2\omega)\hat{u}_\ell(\omega).
          \]

    \item \textbf{Substitute Basis Signals:} Plugging $v = \delta$ ($\hat{v}=1$) and $v = \delta(\cdot-1)$ ($\hat{v}=e^{-i\omega}$) into the PR condition yields the system:
          \[
            \sum_{\ell=0}^s \overline{\hat{\tilde{u}}_\ell(\omega)}\hat{u}_\ell(\omega) + \sum_{\ell=0}^s \overline{\hat{\tilde{u}}_\ell(\omega+\pi)}\hat{u}_\ell(\omega+\pi) = 1,
          \]
          \[
            \sum_{\ell=0}^s \overline{\hat{\tilde{u}}_\ell(\omega)}\hat{u}_\ell(\omega) - \sum_{\ell=0}^s \overline{\hat{\tilde{u}}_\ell(\omega+\pi)}\hat{u}_\ell(\omega+\pi) = 1.
          \]

    \item \textbf{Solve System:} Adding and subtracting these equations gives condition (iv).
  \end{enumerate}
\end{frame}

\begin{frame}{(iv) $\Rightarrow$ (ii) and (ii) $\Rightarrow$ (i)}
  \begin{itemize}
    \item (iv) $\Rightarrow$ (ii): Condition (iv) implies the frequency identity holds for all $v \in \ell_0(\mathbb{Z})$ by linearity of Fourier transform.
    \item (ii) $\Rightarrow$ (i):  {Localization Argument:} For any $v \in \ell(\mathbb{Z})$, truncate to local signal $v_n \in \ell_0(\mathbb{Z})$ using finite support of filters. Apply (ii) to show:
          \[
            v(n) = \frac{1}{2}\sum_{\ell=0}^s [S_{u_\ell} T_{\tilde{u}_\ell} v](n).
          \]
  \end{itemize}
\end{frame}

\end{document}
