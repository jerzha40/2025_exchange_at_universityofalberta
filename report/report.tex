\documentclass{article}
\usepackage{amsmath,amssymb,amsfonts}  % For math symbols and fonts
\usepackage{graphicx}                   % For including images
\usepackage{hyperref}                   % For hyperlinks
\usepackage{cite}                       % For citations

\usepackage{algorithm}
\usepackage{algorithmic}

\usepackage{amsthm}
% Define new theorem-like environments
% Define environments
\theoremstyle{definition} % Non-italicized style
\newtheorem{definition}{Definition}[section]
\newtheorem{exercise}{Exercise}[section]

% Title and author info
\title{2025 University of Alberta Exchange Program Report}
\author{Zhang Jinrui\thanks{alternative email:zhangjr1022@mails.jlu.edu.cn} \\ \texttt{jerryzhang40@gmail.com}}

\date{20250711}  % Empty date; optional, you can also specify a date here

\begin{document}

\maketitle

\begin{abstract}
    In this article, basically everything I have done during
    the exchange program at UoA has been collected
    here.
\end{abstract}

\section{courses}
\subsection{optimal transport by Prf. Pass}
\subsection{option pricing by Prf. Frei}
Some thoughts about the reality using of the
model introduced by Prf. Frei was in [1].

\section{Principal}
The basic principal is to find the vulnerability
of the market. To be more practical and more
mathematically, we are longing for the breaking
of the symmetries. According to the
Efficient-market hypothesis\cite[Hypothesis]{EMH}
we can give out the first and fundamental
definition of this report.

\begin{definition}[EMH means Symmetries]
    \label{def:mydefinition}
    We say a market is completely EMH if and
    only if for every condition $A$, the random
    variable of the logarithmic growth rate $W$
    agrees that $\mathbb{E}_{W\mid A}(W)\doteq 0$.
\end{definition}

Analytically, if we want to prove that a market is
EMH we need to prove for all the condition A
$\mathbb{E}_{W\mid A}(W)\doteq 0$, this is
obviously a very large amount of work to do, and
this is also a very profitless effort(we do all
the strive to find that we can't make money).

Base on this definition, all the experiments which
have been taken or will be taken in the future
are dedicated to find a condition $X$ that
$\mathbb{E}_{W\mid X}(W)>0$ significantly.

\section{Current Roadmap}
\subsection{statistics}
\subsubsection{X is Close Price}
Choose $X=\mathbb{R}$ which is the close price.
\subsubsection{X is the difference between MA and close price}
By intuition, when the price is above the MA,
It should some how comes back, and vice versa.
\subsubsection{statistics for Total Variation and fractals}
This method is about Hurst Exponent,
Fractal Dimension and the Bounded Variation
function model.

standard deviation with moving average.
$S(t,n)$ means start from
$t$ include $n$ data point,
that is
$S(t,n)=\sqrt{\frac{1}{n}\sum_{k=t}^{t+n-1}
        \left(P_k-mean(P)\right)}$
for the Hurst Exponent the $S(n)$ shoud be
stable over time

\subsection{machine learning}
\subsubsection{X is prefix sequence}
To find feature in the prefix sequence. The
Transformer and 1D-CNN is now under
the consideration.
This topic derives experiments including
EXP20250119

\subsubsection{Data process}
Currently method is only short period.
The $W$ in Definition~\ref{def:mydefinition}
is $Lr_t$ which is a very local property.
The maximum and minimum is just in the range
of $3$ time unit. This is way to small period.

We need to some how ignore the local fluctuation,
but get the long term trend and the maximum
and minimum in a meaningful sense in
a specific scale extend.

Here I want to introduce a new method based
on the gradient descent.

The method in the history include the
linear regression by Gauss. And the fellow
mathematician developed some high degree
regression method, include polynomial regression,
logistic regression which is even transfinite.

But I want a more simple and more interpolation
like method.

We use a simplest interpolation, which is
the linear interpolation.

By the classical interpolation theory,
the interpolation point in dimension
$t$ should also be non-uniform, and it
will be determined by the Gauss' method.

Now we can use the power of the gradient
descent.

Suppose we want to interpolate $N$ points
to a period of time series
$P_t,t\in\left[T_s,T_e\right]$.
$\Delta=\{\delta_n|\delta_n=
    (\tau_n,\pi_n)\}_{n=1}^N$.
Then for this regression problem the
parameter set is
$\Delta^*=\Delta\setminus\left\{\tau_1,\tau_N\right\}$ which
$
    \left|
    \Delta^*
    \right|=2\left(N-1\right)
$
The boundary must be fixed, that is
$\tau_1=T_s,\tau_N=T_e$

Initialize the set
$\left\{\tau_n\right\}_{n=1}^N$
to be uniformly space from $T_s$ to $T_e$

We assume
$\forall 1\leq n\leq N-1,\tau_n<\tau_{n+1}$
and for each segment
$\left[\tau_n,\tau_{n+1}\right]$
we do the oridary linear regression
for the points
$\left\{\psi_t=\left(t,P_t\right)|
    \tau_n\leq t\leq\tau_{n+1}\right\}$

Then we can get the coefficient of determination
$\left\{R^2_n\right\}_{n=1}^{N-1}$
for each line segment
$\left[\tau_n,\tau_{n+1}\right]$

Then set the target function
$f({\psi_t}_{t\in\left[T_s,T_e\right]},
    \Delta^*)
    =\sum_{n=1}^{N-1}R^2_n$

We want to optimize the $\Delta^*$
and make the
$f({\psi_t}_{t\in\left[T_s,T_e\right]},
    \Delta^*)$
optimum.

To be more practical, we explicitly define
each variable name.

There is one hyper-parameter $N$, which
defined the number of segments.

The input date to fit is a
$\text{(K,2)-tensor}$
$\psi_k=\left(t_k,P_k\right)$,
$k\in\mathbb{Z},1\leq k\leq K$

Then define a $\text{(N)-tensor}$ $\chi_n$,
$n\in\mathbb{Z},1\leq n\leq N$
Use this $\chi_n$ we can calculate
the length $L_n$ of each segment.
use the softmax $\sigma(\cdot)$
to make the sum of $L_n$
is a const;
Set the data length $\Lambda=t_K-t_1$
Then $L_n=\Lambda\sigma(\chi_n)$
This ensure
$\sum_{n=1}^{N}L_n=\Lambda$

Then $\tau_n=(\sum_{i=1}^{n}L_i)+t_1$
and
$n\in\mathbb{Z},0\leq n\leq N$
$\tau_0=t_1$

Then define a new $\text{(N+1)-tensor}$
$\pi_n$
$n\in\mathbb{Z},0\leq n\leq N$

The implementation is in
\cite[computeGraph]{Data_process}.

The result image is as follow.
\begin{figure}[!ht]
    \centering
    % \includegraphics[width=0.45\textwidth]{figs/DataProcess/N=5.png}
    % \includegraphics[width=0.5\textwidth]{figs/DataProcess/N=10.png}
    \caption{Interpolation N=5 N=10}
\end{figure}

\subsubsection{evolution strategy}
evolution strategy can be viewed as a
zeroth-order gradient descent method.
Now let's define some notations.
\begin{definition}[notations]
    \label{def:notations}
    $\tau$ represents the discrete time of the stock.

    $\Delta\tau$ denotes the discretized time interval.

    $\pi(\tau)$ is the stock price at time $\tau$.

    Suppose at $(t,p)$
    an automlized amount
    of money $\Delta M$ is invested.
    We need to have a strategy when to sell.
    so $P_{S_\theta}(\ln(\frac{\tau}{t}),\ln(\frac{\pi}{p}))$
    is the probability to sell at the state
    $(\ln(\frac{\tau}{t}),\ln(\frac{\pi}{p}))$

    Suppose at $(t,p)$
    an automlized amount
    of money $\Delta M$ is selled.
    We need to have a strategy when to buy.
    so $P_{B_\theta}(\ln(\frac{\tau}{t}),\ln(\frac{\pi}{p}))$
    is the probability to buy at the state
    $(\ln(\frac{\tau}{t}),\ln(\frac{\pi}{p}))$

    all the probability is to make a decision
    whether to buy or sell. but if the balance is
    not enough then we can't buy or sell.
    this situation is the survival conditon that
    also i want the model to learn.

    $S_\theta,B_\theta$ is the
    parameter of the model.
    we can define a FCN network with sigmoid
    function to represent the probability.


\end{definition}

\section{Principal of Unbiased Model}
The long last dream of finding the asymmetry
in market then make money from EMH failure
seems to be too hard to achieve.
So in the other hand, given the EMH hypothesis,
hold true for the market, we can dedicate
ourself to find a Unbiased model.

\begin{definition}[UnbiasedModel]
    \label{def:UnbiasedModel}
    We say a model is Unbiased if and
    only if it work under given condition
    $\mathbb{E}_{W\mid A}(W)\doteq 0$
    \ref{def:mydefinition}.
\end{definition}

This is hard to manage to do, but not all totally
impossible.
\subsection{sochastic differential equation}
This section is a background theory which not
a specific method.
\begin{definition}[autonomous sochastic differential equation]
    \label{def:aSDE}
    $x_0=X(t_0)$ is the initial value.
    $dx_t=V(x_t)dt$
    where $V(x)$ is a random variable
    which determined only by the position
    vector $x_t$
    Then the solution of this equation
    $X(t)=\int_{t_0}^tV(X(s))ds$
    then for each $t$
    we want to get the random distribution of
    $X(t)$ this random distribution is
    the solution to the original equation.
\end{definition}

\subsection{+1s-2s method}
The theory is as follow.
We begin with the random variable in
\ref{def:mydefinition} as a normal distribution
$N(W,\sigma)$ which $\mathbb{E}(W)\doteq 0$

A simple derivation would give us result as follow.
set a lower sell bar $Ls=-2\sigma$,
and a higher sell bar $Hs=+1\sigma$
for the first time step dt we have a distribution
$D_1(x)=N(W,\sigma)$
By the way, $D_0$ is a single point distribution.
and the probability of the first step can
be simplify as a discrete variable $W_1$ which
$$
    \begin{array}{c|ccc}
        action & Ls       & Hold   & Hs     \\
        W_1    & -2\sigma & 0      & \sigma \\
        \hline
        P(W_1) & 0.0228   & 0.8185 & 0.1587 \\
    \end{array}
$$
Which the expect contribution is
$\mathbb{E}_1
    =-2\sigma0.0228+\sigma0.1587
    =0.1131\sigma$

Then we need to calculate the second step
$D_2(x)=
    \frac{\int_{-2\sigma}^{\sigma}D_1(t)D_1(x-t)dt}
    {\int_{-2\sigma}^{\sigma}D_1(t)dt}$
simplify a little bit we can get
$\overline{D_2(x)}=0.8185D_2(x)=
    \int_{-2\sigma}^{\sigma}D_1(t)D_1(x-t)dt$
The discrete variable $W_2$ which
$$
    \begin{array}{c|ccc}
        action & Ls       & Hold   & Hs     \\
        W_2    & -2\sigma & 0      & \sigma \\
        \hline
        P(W_2) & 0.0713   & 0.7544 & 0.1743 \\
    \end{array}
$$
Which the expect contribution is
$\mathbb{E}_2
    =(-2\sigma0.0713+\sigma0.1743)0.8185
    =0.02596\sigma$

Repeat this process we can get
$D_3(x)=
    \frac{\int_{-2\sigma}^{\sigma}D_2(t)D_2(x-t)dt}
    {\int_{-2\sigma}^{\sigma}D_2(t)dt}$
simplify a little bit we can get
$\overline{D_3(x)}=0.7544*0.8185D_2(x)=
    \int_{-2\sigma}^{\sigma}D_2(t)D_2(x-t)dt$
The discrete variable $W_3$ which
$$
    \begin{array}{c|ccc}
        action & Ls       & Hold   & Hs     \\
        W_3    & -2\sigma & 0      & \sigma \\
        \hline
        P(W_3) & 0.1619   & 0.6827 & 0.1554 \\
    \end{array}
$$
Which the expect contribution is
$\mathbb{E}_3
    =(-2\sigma0.1619+\sigma0.1554)0.8185*0.7544
    =-0.1040\sigma$

So we should end sell it within two day.

For further experiments we need to statistically
get the distribution $D_1(W)$ in stead of
assuming it is a normal distribution.

The action strategy is to buy and wait one day.
check if it exceed the $Hs$ or $Ls$.
and then wait for the next day.
and sell any way.
to get a total expectation of
$\mathbb{E}
    =\mathbb{E}_1+\mathbb{E}_2
    =0.1391\sigma$

And notice when in practice
$\mathbb{E}-fee$ should greater than zero, or
it won't beat the cost fee.

Unfortunately,
$\mathbb{E}_1>=0$
could be a result that
$\int_{-2\sigma}^{\sigma}D_2(t)dt<=0$
and
exactly
$\mathbb{E}_1+\int_{-2\sigma}^{\sigma}D_2(t)dt=0$

But this method still have one hope to be
verify by the pure mathematics
is that
the expectation sequence
add up to zero
which means
$\delta_i^i\mathbb{E}_i=0$
but we just need to cut off when
$\mathbb{E}_i<0$ and sell out. it will be fine?

Short gacha guess would reveal
that it will have a expectation of
$\mathbb{E}_{total}=-0.5\sigma<0$

which we need the
$\mathbb{E}_{cut}>0.5\sigma$

\subsection[bsPair]{max draw back control bsPair method}
\begin{figure}[!ht]
    \centering
    % \includegraphics[width=0.45\textwidth]{figs/maxDrawBackControl/Wave.jpg}
    \caption{illustration}
    \label{fig:bsPair}
\end{figure}

Sometimes, the "keep holding" strategy is
to cost in one side, is that, the draw
back is very high, basically, the same
as the local variation rate.
So we need to control the maximum draw back.

This time I want to log a method,
bsPair method.
The basic idea is to sell when there is a
$1\%$ draw back. We sell it, and then,
when the price hit this again, we buy it again.
As Figures\ref{fig:bsPair} shows,
'b' is the buy point, and 's' is the
sell point.
suppose the trade fee is $\phi$.
And the random variable $N_{\phi}$ is the
total pair number of buy and sell,
after the case is over.
I say a case is over is the automata fufill
the sell and finished bar such as reach the
level $P_H$.
Then the expectation of the random variable
is $\mathbb{E}(N_{\phi})$.

Suppose the price is $P_{t_0}$ at time $t_0$,
we buy $\mu$ amount of stock at time $t_0$,
The two line is $P_L$ and $P_H$.

Normally in automata we have
$C_H=\ln(\frac{P_H}{P_{t_0}})$
and
$C_L=\ln(\frac{P_L}{P_{t_0}})$
is fixed, to inversely get $P_L$ and $P_H$.

$\mu$ amount of stock is the principal.
And the total gas fee is
$\phi\mathbb{E}(N_{\phi})$,
By the time we sell at $P_H$.

So all the fee is $\phi\mathbb{E}(N_{\phi})$
the total profit is
$\mu\frac{P_H}{P_{t_0}}=\mu\exp(C_H)$.

And the fee rate is $C_\phi$
which $\phi=\mu\exp(C_\phi)$

We need
$\mu\exp(C_\phi)\mathbb{E}(N_{\phi})
    \ll
    \mu\exp(C_H)$

Which is equivalent to
$C_\phi+\ln(\mathbb{E}(N_{\phi}))
    \ll
    C_H$

When $C_\phi\doteq5\times10^{-3}$
and
$C_H\doteq10\times10^{-3}$
we can have total logarithmic growth
rate at
$C=
    \ln(
    \exp(C_H)-\exp(C_\phi)\mathbb{E}(N_{\phi}))$.

When in classic automata, we have
$\mathbb{E}(N_{\phi})=2$

But if we have this method, we can control
the maximum draw back, which is
estimated as
$\chi=
    \mu
    (\exp(C_\phi)\mathbb{E}(N_{\phi})
    +
    \exp(C_L))$

So we can take back $1-\chi$
of the principal.

If the first condition
$\mu\exp(C_\phi)\mathbb{E}(N_{\phi})
    \ll
    \mu\exp(C_H)$
is satisfied, then we can
estimated
$\exp(C_\phi)\mathbb{E}(N_{\phi})
    +
    \exp(C_L)<\exp(C_L)+\exp(C_H)$

\begin{definition}[max draw back]
    \label{def:maxDrawBack}
    The max draw back is the max draw back
    because of the variation of the price.
\end{definition}

\section{Current Roadmap of UnbiasedModel}
\subsection{Grids}
Grids are a very old and simple method and
there is nothing special to say about this
method.

But there are several experiments need to
be carried out to get a better optimization
of this ancient method.

First we need to get the distribution of the
$P_t$
We always refered to the distribution of this
$Lr_t=\ln(\frac{P_{t+1}}{P_{t}})$
But now we want the distribution of $P_t$
itself.

$P_t$ is the historical data which is definite
set the random variable $\rho$ is the random
variable of $P_t$
The meaning is you get a time you can say the
possibility of get $P$ is $P(\rho=P)$


\section{EXP20250119 prefix sequence}
This experiment use the prefix sequence
to be the condition space X.

\begin{definition}[prefix space]
    All possible prefix sequence form a space X
\end{definition}

If the market is EMH.
By Definition~\ref{def:mydefinition}, this means
for all subset of $X$,
$\mathbb{E}_{W\mid A}(W)\doteq 0$. That is
$
    \forall A\subseteq X,
    \mathbb{E}_{W\mid A}(W)\doteq 0
$

So if we want to find the vulnerability of
the market we want to find a $X$ such that
there exist a $A\subseteq X$ agrees that
$\mathbb{E}_{W\mid A}(W)>0$

Consider a simplest senario, that is, we only
treat the sequence as descend and ascend. Then
for a space of length $L$ then
$\left|X\right|=2^L$, which is a nightmare for
statistics, so I'm forced to use machine learning
to try to find something.

\subsection{Data regularization}
Current data regularization is as follow.
Suppose we have a time series
$P_t \in \mathbb{R}_+$
We want it range from $(0,1)$ and have a symmetry
about descend and ascend around zero.
So let $Lr_t=\ln(\frac{P_{t+1}}{P_{t}})$.
$Lr_t\in \mathbb{R}$ and then
$Sg_t=\tanh(Lr_t)$

Then $X=\{(Sg_k)|t_{start}\leq k\leq t_{end}\}$

\subsection{Conclusion}
In \cite[PrefixSequence]{EXP20250119PrefixSequence}
I implemented the function to get $Sg_t$

The first Stage experiment result is
very bad in \cite[Cov1d]{EXP20250119PrefixSequenceGPTCov1d}

\section{EXP20250120 MA}
This experiment use the difference between
MA and close price as the space X.

\begin{definition}[MACP difference]
    $D_t^N:=A_t^N-P_t$
    which
    $A_t^N=\sum_{k=t-N+1}^tP_k$
    which $N$ is a hyper-parameter
    when N is a single number then $D_t:=D_t^N$
    is a time series of number
    when $N$ can be multiple different numbers
    then $D_t^N$is a vector time series.
\end{definition}

\subsection[Fix N]{Fix $N$}
Suppose $N$ is a fixed single number then
$D_t$ is a time serie.
To simplify the problem we define a
classification variable $V$
\begin{definition}[Ternary Classification]
    $
        V_t:=
        \begin{cases}
            1,  & P_t\in\left(0.1,+\infty\right)  \\
            0,  & P_t\in\left(-0.1,0.1\right)     \\
            -1, & P_t\in\left(-\infty,-0.1\right) \\
        \end{cases}
    $
\end{definition}

Use the bin size of $0.01$ which is the smallest
increment of the price. Do the statistics over
all $D_t\in\mathbb{R}$ with the bin size $0.01$
to obtain the conditional probability distribution
of the random variable $V_t$

\subsection{Result}
\begin{figure}[!ht]
    \centering
    % \includegraphics[width=0.45\textwidth]{figs/EXP20250120MA/EXP20250120MA_RandomWalk.png}
    % \includegraphics[width=0.5\textwidth]{figs/EXP20250120MA/EXP20250120MA_PRICE.png}
    \caption{RandomWalk(left),realdata(right)}
\end{figure}
The result shows for the Random RandomWalk, the
curve is like sigmoid but exchude a normal
distribution for $V_t=0$
While the distribution for realdata is more
intricate the blue curve is more like a
t-distribution.
\begin{figure}[!ht]
    \centering
    % \includegraphics[width=0.45\textwidth]{figs/EXP20250120MA/EXP20250120MA_Prc.png}
    % \includegraphics[width=0.5\textwidth]{figs/EXP20250120MA/EXP20250120MA_Prc_.png}
    \caption{Fixed:RandomWalk(left),realdata(right)}
\end{figure}

The original code that GPT write, use $D_{t+1}$
to predict $V_t$ which in the RandomWalk shows
significantly distribution bias.This fixed version
has test on the RandomWalk don't have any bias
on the RandomWalk.


\section{EXP20250615 +1s-2s method}
This experiment would practice the
+1s-2s method.
Basically it needs to statistically get
the distribution of the random variable
$D_1$ and to compute the convolution
cross verify the result of $D_2$
and get the expectation.

\subsection[statistic D1]{statistic of $D_1$}

\section[EXP20250702 ρ]{EXP20250702 $\rho$}
Suppose we have the historical data $P_t$
We want to get $\rho$.
We need a function to get the distribution
of the $\rho$, Like a box-line graph, of any
given time period.



\bibliographystyle{plain}  % or another style like unsrt, IEEEtran, etc.
\bibliography{references}  % references.bib is the file name

\end{document}
