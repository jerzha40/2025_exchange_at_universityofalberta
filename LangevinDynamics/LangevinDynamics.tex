\documentclass{article}
\usepackage{amsmath,amssymb,amsfonts}  % For math symbols and fonts
\usepackage{graphicx}                   % For including images
\usepackage{hyperref}                   % For hyperlinks
\usepackage{cite}                       % For citations

\usepackage{algorithm}
\usepackage{algorithmic}

\usepackage{amsthm}
% Define new theorem-like environments
% Define environments
\theoremstyle{definition} % Non-italicized style
\newtheorem{definition}{Definition}[section]
\newtheorem{exercise}{Exercise}[section]

% Title and author info
\title{A sample method for overdamped Langevin Equation}
\author{Zhang Jinrui\thanks{alternative email:zhangjr1022@mails.jlu.edu.cn} \\ \texttt{jerryzhang40@gmail.com}}

\date{20250714}  % Empty date; optional, you can also specify a date here

\begin{document}

\maketitle

\begin{abstract}
    In this article, I tried to get the solution
    distribution of the overdamped Langevin Equation.
\end{abstract}

\section{recite of the problem}
The overdamped Langevin Equation use the following
version from wikipedia
\cite[LangevinDynamics]{LangevinDynamics}
\[
    \mathrm{d}\mathbf{X} = -\frac{1}{\gamma} \nabla U(\mathbf{X}) \,\mathrm{d}t + \frac{\sqrt{2}\sigma}{\gamma} \,\mathrm{d}\mathbf{W}(t)
\]
I choose to solve this one for some simplisity.
\[
    \mathrm{d}\mathbf{X} = \nabla U(\mathbf{X}) \,\mathrm{d}t +  \,\mathrm{d}\mathbf{W}(t)
\]
or this one.
\[
    \dot{\mathbf{X}} =  \nabla U(\mathbf{X}) +  \,\boldsymbol{\eta}(t)
\]
where \(\eta=\dot{\mathbf{W}}\)
\[
    \langle \eta_i(t) \rangle = 0, \quad
    \langle \eta_i(t) \eta_j(t') \rangle = \delta_{ij} \delta(t - t')
\]

\section{roadmap}
The main idea to solve this equation is
to use the pinn\cite{raissi2017physics} to fitting the solution
distribution \(p(x,t)\)
where for every \(t\) we have
\(\int_{-\infty}^{\infty}p(x,t)dx=1\)

We need a differential equation about this
\(p(x,t)\) to make the pinn work.

First we need to find a differential
relation only involves \(p(x,t)\).
Then we can use pinn to get the solution
distribution \(p(x,t)\).

\subsection{definite equation}
To simplify the original equation
\[
    \dot{\mathbf{X}} =  \nabla U(\mathbf{X}) +  \,\boldsymbol{\eta}(t)
\]
We first remove the wenier term,
as following.
\[
    \dot{\mathbf{X}} =  \nabla U(\mathbf{X})=u
\]

For a short period of time increment \(dt\)
the problem involves two distribution
\(p(x,t)\) and \(p(x,t+dt)\)
and we choose a initial point at \(t\)
denote as \(x_t\) and the solution point
at time \(t+dt\) is \(x_{t+dt}\)

So the Lagrangian derivative
is as follow
\[
    \frac{Dp}{Dt} = \frac{\partial p}{\partial t} + u \cdot \nabla p
\]
\[
    \frac{Dp}{Dt}=\frac{p(x_{t+dt},t+dt)-p(x_{t},t)}{dt}
\]
and Consider the probabilistic mass around the
point \(x_{t+dt}\) and \(x_{t}\) we can get a following
relation
\[p(x_{t+dt},t+dt)(dx)(e^{vdt})=p(x_{t},t)(dx)\],
where \[v=\nabla \cdot u\]
which is the same as follow
\[p(x_{t+dt},t+dt)=p(x_{t},t)(e^{-vdt})\]
as the \(dt\) is infinitesimal so
\(e^{-vdt}=1-vdt\)
so we can get
\[
    p(x_{t+dt},t+dt)-p(x_{t},t)=-vdt
\]
which is
\[
    \frac{Dp}{Dt}=-v=-\nabla \cdot u=-\Delta U=\frac{\partial p}{\partial t} + u \cdot \nabla p
\]
then we get a formula only involves \(p(x,t)\)
\[
    \frac{\partial p}{\partial t} + u \cdot \nabla p +\nabla \cdot u=0
\]
the initial distribution may take any thing,
but for one case we can chose the dirac function
\[
    p(x,0)=\delta(x - x_0)
\]
in this case this model is just a plain
ODE. but in a distribution case.

\subsection{sochastic equation}
Add back the wenier term the equation backs to
\[
    \dot{\mathbf{X}} =  \nabla U(\mathbf{X}) +  \,\boldsymbol{\eta}(t)
\]
For the same period of time increment \(dt\)
the problem involves two distribution
\(p(x,t)\) and \(p(x,t+dt)\)
and we choose a initial point at \(t\)
denote as \(x_t\) and the solution point
at time \(t+dt\) is \(x_{t+dt}\)

then for the point relations by the solution of the
equation, \(x_t\) gose to \(x_{t+dt}\)
we denote this as \(x_{t+dt}=F(x_t)\)
and \(x_{t}=B(x_{t+dt})\)

For a certain point \(x\) at time \(t+dt\)
we want \(p(x,t+dt)\)
this involves a convolution process.
\[
    p(x,t+dt)=\int_{-\infty}^{\infty}g(y)p(B(x),t)e^{-vdt}dy
\]
where $g(y)$ is the probabilistic density function of
normal distribution
\[
    \mathcal{N}(0, dt)
\]
If we can some how transform this equation to a
differential equation only involves
\(p(x,t)\), then we can solve this
by pinn as the definite model.
But this process is way too hard to deal.

\subsection{sochastic equation:sampling}
I suppose to sample the
\[
    \boldsymbol{\eta}(t)
\]
According to the time discrete size when we
use pinn.
when we have sample a specific sample of
\[
    \boldsymbol{\eta}_k(t)
\]
we can then rewrite
\[
    \dot{\mathbf{X}} =  u_t=\nabla U(\mathbf{X}) +  \,\boldsymbol{\eta}(t)
\]
\[
    \frac{\partial p}{\partial t} + u_t \cdot \nabla p +\nabla \cdot u_t=0
\]
sove this equaiton we can get a sample solution
\[
    p_k(x,t)
\]
to sample a lot
\[
    \boldsymbol{\eta}_k(t)
\]
we can average all the \(p_k\)
to get the estimated distribution function
\(p\sim\frac{\sum_{k=1}^{N}p_k}{N}\)









\bibliographystyle{plain}  % or another style like unsrt, IEEEtran, etc.
\bibliography{references}  % references.bib is the file name

\end{document}
