\documentclass[12pt, a4paper, oneside]{ctexart}
\usepackage{amsmath, amsthm, amssymb, appendix, bm, graphicx, hyperref, mathrsfs,geometry,mathtools,esint}
\pagestyle{plain}
\title{\textbf{论文标题}}
\author{Dylaaan}
\date{\today}
\linespread{1.5}
\newtheorem{theorem}{定理}[section]
\newtheorem{definition}[theorem]{定义}
\newtheorem{lemma}[theorem]{引理}
\newtheorem{corollary}[theorem]{推论}
\newtheorem{example}[theorem]{例}
\newtheorem{proposition}[theorem]{命题}
\renewcommand{\abstractname}{\Large\textbf{摘要}}
\geometry{
	left=2cm,
	right=2cm,
	top=2cm,
	bottom=2cm,
	% 如果需要设置页眉页脚的高度,可以添加以下参数
	% headheight=2cm,
	% headsep=1cm,
	% footskip=1cm,
}
\allowdisplaybreaks[4]
\begin{document}
考虑带阻力的动力方程
\begin{equation*}
	m\frac{\mathrm{d}^2\boldsymbol{p}_i}{\mathrm{d}t^2}+c\left\|\frac{\mathrm{d}\boldsymbol{p}_i}{\mathrm{d}t}\right\|\frac{\mathrm{d}\boldsymbol{p}_i}{\mathrm{d}t}=\boldsymbol{u}_i,
\end{equation*}
其中 $m$ 为质量, $c$ 为阻力系数, 即
\begin{align}
	\frac{\mathrm{d}\boldsymbol{X}}{\mathrm{d}t}=\boldsymbol{F}(\boldsymbol{X}),
\end{align}
其中
\begin{equation*}
	\boldsymbol{X}=\binom{\boldsymbol{p}}{\boldsymbol{v}}\in\mathbb{R}^{40},\ \boldsymbol{p}=\begin{pmatrix}
		\boldsymbol{p}_1\\\vdots\\\boldsymbol{p}_{10}
	\end{pmatrix},\ \boldsymbol{v}=\begin{pmatrix}
	\boldsymbol{v}_1\\\vdots\\\boldsymbol{v}_{10}
\end{pmatrix},\ \boldsymbol{F}(\boldsymbol{X})=\binom{\boldsymbol{v}}{m^{-1}(\boldsymbol{u}-c\|\boldsymbol{v}\|\boldsymbol{v})}.
\end{equation*}
易知质心 $\boldsymbol{c}(\boldsymbol{X}),\|\boldsymbol{r}_i\|,\widehat{\boldsymbol{r}_i},\mathit{\Phi}$ 均为 Lipschitz 连续函数, 且初值条件满足
\begin{equation*}
	\min_{i\ne j}\|\boldsymbol{p}_i(0)-\boldsymbol{p}_j(0)\|>d_s>0.
\end{equation*}
由 Picard 定理可知方程 (1) 的解存在且唯一, 解为
\begin{equation*}
	\boldsymbol{X}(t)=\boldsymbol{X}(0)+\int_0^t\boldsymbol{F}(\boldsymbol{X}(s))\mathrm{d}s,
\end{equation*}
即
\begin{align*}
	&\boldsymbol{v}_i(t)=\boldsymbol{v}_i(0)+\int_0^t\left[m^{-1}\boldsymbol{u}_i(\boldsymbol{X}(s))-\frac{c}{m}\|\boldsymbol{v}_i(s)\|\boldsymbol{v}_i(s)\right]\mathrm{d}s,\\
	&\boldsymbol{p}_i(t)=\boldsymbol{p}_i(0)+\int_0^t\boldsymbol{v}_i(s)\mathrm{d}s.
\end{align*}
下面证明防撞性. 定义
\begin{align*}
	&\mathit{\Psi}(d)=k_r\exp\left\{-\frac{(d-d_s)^2}{2\sigma^2}\right\},\\
	&\mathit{\Phi}(d)=-\frac{\mathrm{d}\mathit{\Psi}}{\mathrm{d}d}=\frac{k_r}{\sigma^2}\exp\left\{-\frac{(d-d_s)^2}{2\sigma^2}\right\}(d-d_s),\\
	&E_{ij}(t)=\frac{1}{2}\left(\frac{\mathrm{d}d_{ij}}{\mathrm{d}t}\right)^2+\psi(d_{ij}(t)),
\end{align*}
其中 $d_{ij}(t)=\|\boldsymbol{p}_i(t)-\boldsymbol{p}_j(t)\|.$ 则系统能量为
\begin{equation*}
	\mathcal{E}(t)=\sum_{1\leqslant i<j\leqslant N}E_{ij}(t).
\end{equation*}
而
\begin{equation*}
	\frac{\mathrm{d}E_{ij}}{\mathrm{d}t}=\frac{\mathrm{d}d_{ij}}{\mathrm{d}t}\cdot\frac{\mathrm{d}^2d_{ij}}{\mathrm{d}t^2}+\mathit{\Phi}(d_{ij})\frac{\mathrm{d}d_{ij}}{\mathrm{d}t},
\end{equation*}
\begin{equation}
	\frac{\mathrm{d}^2d_{ij}}{\mathrm{d}t^2}=\frac{1}{d_{ij}}\left[\|\boldsymbol{v}_i-\boldsymbol{v}_j\|^2+(\boldsymbol{p}_i-\boldsymbol{p}_j)\cdot(\boldsymbol{u}_i-\boldsymbol{u}_j)-\left(\frac{\mathrm{d}d_{ij}}{\mathrm{d}t}\right)^2\right],
\end{equation}
其中 $\boldsymbol{u}_i=\dot{\boldsymbol{v}_i}.$ 对于 $\boldsymbol{u}_i$, 有
\begin{align*}
	\boldsymbol{u}_i=\frac{1}{m}[-k_p(d_i-R^*)\widehat{\boldsymbol{r}_i}-k_dv_{r,i}\widehat{\boldsymbol{r}_i}-k_v(v_{\theta,i}-v_d)\hat{\boldsymbol{\theta_i}}]+\frac{1}{m}\sum_{k\ne i}\mathit{\Phi}(d_{ik})(\boldsymbol{p}_i-\boldsymbol{p}_k)=:\boldsymbol{u}_{i_1}+\boldsymbol{u}_{i_2}.
\end{align*}
设 $\|\boldsymbol{u}_{i_1}-\boldsymbol{u}_{j_1}\|\leqslant L.$ 取
\begin{equation*}
	k_r>L\sigma^2\mathrm{e}^{\frac{1}{2}}\max\left\{\frac{1}{d_s},\frac{1}{\displaystyle\min_{k\ne l}d_{kl}(0)}\right\},
\end{equation*}
则当 $d_{ij}\leqslant d_s+\sigma$ 时, 有
\begin{equation*}
	\|\boldsymbol{u}_{i_2}-\boldsymbol{u}_{j_2}\|>2L.
\end{equation*}
于是
\begin{equation*}
	(\boldsymbol{p}_i-\boldsymbol{p}_j)\cdot(\boldsymbol{u}_i-\boldsymbol{u}_j)\geqslant\|\boldsymbol{u}_{i_2}-\boldsymbol{u}_{j_2}\|d_{ij}-Ld_{ij}>Ld_{ij}.
\end{equation*}
代入 (2) 式有
\begin{equation*}
	\frac{\mathrm{d}E_{ij}}{\mathrm{d}t}\geqslant\frac{\mathrm{d}d_{ij}}{\mathrm{d}t}(L+\mathit{\Phi}(d_{ij}))>0.
\end{equation*}
由此即知
\begin{equation*}
	\frac{\mathrm{d}\mathcal{E}}{\mathrm{d}t}\geqslant-\kappa\mathcal{E}(t),
\end{equation*}
其中 $\kappa>0$ 为常数. 而
\begin{align*}
	&\mathcal{E}(0)\geqslant\sum_{i<j}\mathit{\Psi}(d_{ij}(0))>\psi(d_s+\sigma)\cdot\binom{N}{2},\\
	&\mathcal{E}(t)\geqslant\mathcal{E}(0)\mathrm{e}^{-\kappa t}>0,\\
	&\mathit{\Psi}(d_{ij}(t))\leqslant E_{ij}(t)\leqslant\mathcal{E}(t).
\end{align*}
由 $\mathit{\Psi}$ 严格单调递减可知
\begin{equation*}
	d_{ij}(t)\geqslant\mathit{\Psi}^{-1}(\mathcal{E}(t))>\mathit{\Psi}^{-1}(\mathcal{E}(0)\mathrm{e}^{-\kappa t}).
\end{equation*}
令 $t\to\infty,$ 有
\begin{equation*}
	\varliminf_{t\to\infty}d_{ij}(t)\geqslant\mathit{\Psi}^{-1}(0)=d_s.
\end{equation*}
故总是不会相撞.\par
下面讨论收敛性, 即讨论系统收敛至
\begin{equation*}
	\mathcal{S}=\{\|\boldsymbol{r}_i\|=R,\boldsymbol{v}_i\cdot\widehat{\boldsymbol{r}_i}=0,\|\boldsymbol{v}_i\|=v_d\}.
\end{equation*}
构造 Lyapunov 函数
\begin{equation*}
	V=\frac{1}{2}\sum_{i=1}^N[k_p(d_i-R)^2+\|\boldsymbol{v}_i-v_d\widehat{\boldsymbol{\theta}_i}\|^2]+\sum_{i<j}\mathit{\Psi}(d_{ij}),
\end{equation*}
则
\begin{equation*}
	\dot{V}=-\sum_{i}k_dv_{r,i}^2-\sum_ik_v(v_{\theta,i}-v_d)^2-\sum_ic\|\boldsymbol{v}_i\|^3\leqslant0,
\end{equation*}
故方程渐进收敛至 $\mathcal{S}$.
\end{document}






















